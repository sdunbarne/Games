%%% -*-LaTeX-*-
%%% games.tex.orig
%%% Prettyprinted by texpretty lex version 0.02 [21-May-2001]
%%% on Wed May 18 10:43:23 2022
%%% for Steve Dunbar (sdunbar@family-desktop)

\documentclass[12pt]{article}

\input{../../../../etc/macros} %% \input{../../../../etc/mzlatex_macros}
\input{../../../../etc/pdf_macros}

\bibliographystyle{plain}

\setcounter{MaxMatrixCols}{14} %accommodate large matrices

\begin{document}

\myheader \mytitle

\hr

\sectiontitle{Games as Markov Chains}

\hr

\usefirefox

\hr

% \visual{Study Tip}{../../../../CommonInformation/Lessons/studytip.png}
% \section*{Study Tip}

% \hr

\visual{Rating}{../../../../CommonInformation/Lessons/rating.png}
\section*{Rating} %one of
% Everyone: contains no mathematics.
% Student: contains scenes of mild algebra or calculus that may require guidance.
Mathematically Mature:  may contain mathematics beyond calculus with
proofs.  % Mathematicians Only: prolonged scenes of intense rigor.

\hr

\visual{Section Starter Question}{../../../../CommonInformation/Lessons/question_mark.png}
\section*{Section Starter Question} What feature of a board game is
necessary for it to be modeled as a Markov chain?  Can you think of a
board game that can be modeled as a Markov chain?

\hr

\visual{Key Concepts}{../../../../CommonInformation/Lessons/keyconcepts.png}
\section*{Key Concepts}

\begin{enumerate}
    \item
        \emph{Tenzi} is a simple dice game for multiple players.  The
        gameplay of Tenzi for a single player can be modeled with a
        Markov chain with \( 10 \) transient states and \( 1 \)
        absorbing state.  Markov chain methods give the distribution of
        the number of rolls necessary to complete the game, along with
        the mean and variance.
    \item
        The \emph{Towers of Hanoi} is a mathematical game or puzzle.  It
        consists of three rods and some number of disks of different
        sizes, with holes in the center so each disk can slide onto any
        rod.  The puzzle starts with the disks in a neat stack in
        ascending order of size on one rod, the smallest at the top,
        making a conical shape.  The objective of the puzzle is to move
        the entire stack to another rod, obeying \( 3 \) rules.
    \item
        The childrens' game \emph{Chutes and Ladders} is a \( 100 \)
        state Markov chain with a final absorbing state and Markov chain
        methods give the expected playing time.
    \item
        \emph{Count Your Chickens!} is a cooperative counting game aimed
        at preschoolers.  Absorbing Markov chains give the probability
        of winning a game of \emph{Count Your Chickens!} and the
        expected number of chicks at the end of the game.  The main
        interest is classifying and counting the states, leading to a \(
        165 \times 165 \) transition probability matrix.
\end{enumerate}

\hr

\visual{Vocabulary}{../../../../CommonInformation/Lessons/vocabulary.png}
\section*{Vocabulary}
\begin{enumerate}
    \item
        \emph{Tenzi} is a simple dice game for multiple players.
    \item
        The \emph{Towers of Hanoi} is a mathematical game or puzzle.  It
        consists of three rods and some number of disks of different
        sizes, with holes in the center so each disk can slide onto any
        rod.
    \item
        The childrens' game \emph{Chutes and Ladders} is a board game
        that can be modeled as a \( 100 \) state Markov chain with a
        final absorbing state.
    \item
        \emph{Count Your Chickens!} is a cooperative counting board game
        aimed at preschoolers.
\end{enumerate}

\hr

\section*{Notation}
\begin{enumerate}
    \item
        \( i \), \( j \) -- typical states in a game Markov chain
    \item
        \( P(j) \) -- cumulative probability function for a player
        getting Tenzi on or before the \( j \)th roll
    \item
        \( p(j)=P(j) - P(j - 1) \) -- probability density function for a
        player getting Tenzi on the \( k \)th roll
    \item
        \[
            P =
            \begin{pmatrix}
                Q & R \\
                0 & I
            \end{pmatrix}
        \] transition probability matrix is already partitioned
    \item
        \( N \) matrix of expected number of rolls a player will spend
        in each state
    \item
        \( m \) -- number of disks in Tower of Hanoi problem
    \item
        \( aaa, aba, \dots \) -- representation of the states for \( 3 \)
    \item
        \( (a, b) \) -- A state in \emph{Count Your Chickens!} as an
        ordered pair where \( a \) is the mama chicken's position on the
        board and \( b \) is the number of chicks in the coop
    \item
        \( b_i^R \) -- the number of blue squares after square \( i \)
        on the path and
    \item
        \( b_i^L \) -- the number of blue squares that come before
        square \( i \), including square \( i \)
\end{enumerate}

\visual{Mathematical Ideas}{../../../../CommonInformation/Lessons/mathematicalideas.png}
\section*{Mathematical Ideas}

This section examines \( 4 \) additional games as Markov chains.  The
goal is to determine the expected number of moves or plays to reach a
winning state.  The games have from \( 11 \) states to \( 165 \) states
and another goal is to show how to construct the states and the
transition probability matrices in these cases.

\subsection*{Tenzi}

\emph{Tenzi} is a simple dice game for multiple players.  Each player
has \( 10 \) six-sided fair dice in their hands.  Taking turns, one
player calls out, e.g., ``Tenzi Four!'' Everyone starts rolling their
dice as quickly as possible.  Whenever a ``four'' comes up, the player
sets it aside and rolls the remaining dice.  Eventually the player
achieves all \( 10 \) ``fours'' and calls out ``Tenzi!'', becoming the
winner of the round.%
\index{\emph{Tenzi}}

A Markov chain models the gameplay of \emph{Tenzi} for a single player.
Each state in the Markov chain represents the player's accumulated
number of dice matching the target number.  So at any point in the game,
a player will be in one of the \( 11 \) states \( \set{0,1,2,3, \ldots,
10} \).  A player in state \( 0 \) has not yet rolled any dice matching
the target number, and a player in state \( 10 \) will call out
``Tenzi!' Each time a player rolls the dice, the player has a chance to
remain in the current state or to move up to a higher-numbered state.

Each entry in the transition probability matrix is a binomial
probability.  A player in state \( i \) rolls the remaining \( 10 - i \)
dice.  The player moves up to state \( j > i \) if exactly \( j - i \)
of the dice match the target number.  So, the probability of
transitioning from state \( i \) to state \( j \) in one roll of the
dice is \( \binom{10 - i}{j - i}\left( \frac{1}{6} \right)^{j - i} \left
( \frac{5}{6} \right)^{10 - j} \) for \( 0 \le i \), \( j \le 10 \) and \(
i \le j \).  A player will never go backwards to a previous state in the
game, so the probability is \( 0 \) for \( i>j \).  Thus the states are
not communicating, the Markov Chain is not recurrent, states \( 0 \)
through \( 9 \) are transient and the state \( 10 \) is an absorbing
state.~%
\index{binomial probability}

The full \( 11 \times 11 \) transition probability matrix \( P \) is
approximately
\begin{tiny}
    \[
        \begin{pmatrix}
            0.162 & 0.323 & 0.291 & 0.155 & 0.0543 & 0.013 & 0.00217 &
            2.48 \cdot {10^{-4}} & 1.86 \cdot {10^{-5}} & 8.27 \cdot {10^
            {-7}} & 1.65 \cdot {10^{-8}}\\
            0.0 & 0.194 & 0.349 & 0.279 & 0.13 & 0.0391 & 0.00781 &
            0.00104 & 8.93 \cdot {10^{-5}} & 4.47 \cdot {10^{-6}} & 9.92
            \cdot {10^{-8}}\\
            0.0 & 0.0 & 0.233 & 0.372 & 0.26 & 0.104 & 0.026 & 0.00417 &
            4.17 \cdot {10^{-4}} & 2.38 \cdot {10^{-5}} & 5.95 \cdot {10^
            {-7}}\\
            0.0 & 0.0 & 0.0 & 0.279 & 0.391 & 0.234 & 0.0781 & 0.0156 &
            0.00188 & 1.25 \cdot {10^{-4}} & 3.57 \cdot {10^{-6}}\\
            0.0 & 0.0 & 0.0 & 0.0 & 0.335 & 0.402 & 0.201 & 0.0536 &
            0.00804 & 6.43 \cdot {10^{-4}} & 2.14 \cdot {10^{-5}}\\
            0.0 & 0.0 & 0.0 & 0.0 & 0.0 & 0.402 & 0.402 & 0.161 & 0.0322
            & 0.00322 & 1.29 \cdot {10^{-4}}\\
            0.0 & 0.0 & 0.0 & 0.0 & 0.0 & 0.0 & 0.482 & 0.386 & 0.116 &
            0.0154 & 7.72 \cdot {10^{-4}}\\
            0.0 & 0.0 & 0.0 & 0.0 & 0.0 & 0.0 & 0.0 & 0.579 & 0.347 &
            0.0694 & 0.00463\\
            0.0 & 0.0 & 0.0 & 0.0 & 0.0 & 0.0 & 0.0 & 0.0 & 0.694 &
            0.278 & 0.0278\\
            0.0 & 0.0 & 0.0 & 0.0 & 0.0 & 0.0 & 0.0 & 0.0 & 0.0 & 0.833
            & 0.167\\
            0.0 & 0.0 & 0.0 & 0.0 & 0.0 & 0.0 & 0.0 & 0.0 & 0.0 & 0.0 &
            1.0
        \end{pmatrix}
        .
    \]
\end{tiny}
The initial distribution is \( X_0 = (1,0,0,0,0,0,0,0,0,0,0) \).  The
value in the last place of \( X_k \) is the probability of reaching the
absorbing or winning state for Tenzi in \( k \) or fewer rolls.  For
example, the entry in the last entry of \( X_{20} \) is \( 0.7677 \).
The player has a \( 76.77\% \) chance the game finishes by the \( 20 \)th
roll.  The player may have finished on the \( 20 \)th roll, or possibly
finished on some previous roll and is waiting in absorbing state \( 10 \).
The resulting cumulative probability function \( P(j) \) for a player
getting Tenzi on or before the \( j \)th roll is in Figure~%
\ref{fig:games:tenzicdf}.  To find \( p(j) \), the probability of
getting Tenzi in exactly \( j \) rolls, subtract the probability that
the player already had Tenzi by the \( j - 1 \) roll.  This gives the
probability density function \( p(j)=P(j) - P(j - 1) \) also shown in
Figure~%
\ref{fig:games:tenzicdf}.
\begin{figure}
    \centering
    \includegraphics[scale=0.33]{tenzi}
    \caption{Probability density function (in blue) and cumulative
    distribution function (in red) of waiting time to win in Tenzi.}%
    \label{fig:games:tenzicdf}
\end{figure}
The maximum of this probability function is at \( j=13 \); a player has
a \( 0.0705 \) probability of getting Tenzi on the \( 13 \)th roll.
Using these probabilities, the mean of the distribution is \( \sum_{j=1}^
{\infty} j p(j) \).  Approximate this sum through \( j=50 \) as \( 16.5 \).
This is the average number of rolls needed for a player to get Tenzi.

The fundamental matrix provides another way of finding the mean.~%
\index{fundamental matrix}
The transition probability matrix is already partitioned as
\[
    P =
    \begin{pmatrix}
        Q & R \\
        0 & I
    \end{pmatrix}
\] where \( Q \) is the \( 10 \times 10 \) matrix containing the
transition probabilities among the transient states, \( R \) is a \( 10
\times 1 \) column matrix containing the probabilities of going from
each of the non-absorbing states to state \( 10 \) in one roll, \( 0 \)
is a \( 1 \times 10 \) zero matrix, and \( I \) is the \( 1 \times 1 \)
identity matrix representing the absorbing state \( 10 \).  Then \( N \)
is approximately
\[
    \begin{pmatrix}
        1.193 & 0.478 & 0.669 & 0.787 & 0.915 & 1.097 & 1.371 & 1.828 &
        2.742 & 5.485 \\
        0.000 & 1.240 & 0.564 & 0.771 & 0.917 & 1.097 & 1.371 & 1.828 &
        2.742 & 5.485 \\
        0.000 & 0.000 & 1.303 & 0.673 & 0.905 & 1.099 & 1.371 & 1.828 &
        2.742 & 5.485 \\
        0.000 & 0.000 & 0.000 & 1.387 & 0.815 & 1.091 & 1.373 & 1.828 &
        2.742 & 5.485 \\
        0.000 & 0.000 & 0.000 & 0.000 & 1.504 & 1.010 & 1.368 & 1.829 &
        2.742 & 5.485 \\
        0.000 & 0.000 & 0.000 & 0.000 & 0.000 & 1.672 & 1.298 & 1.826 &
        2.743 & 5.485 \\
        0.000 & 0.000 & 0.000 & 0.000 & 0.000 & 0.000 & 1.931 & 1.769 &
        2.742 & 5.485 \\
        0.000 & 0.000 & 0.000 & 0.000 & 0.000 & 0.000 & 0.000 & 2.374 &
        2.697 & 5.485 \\
        0.000 & 0.000 & 0.000 & 0.000 & 0.000 & 0.000 & 0.000 & 0.000 &
        3.273 & 5.455 \\
        0.000 & 0.000 & 0.000 & 0.000 & 0.000 & 0.000 & 0.000 & 0.000 &
        0.000 & 6.000 \\
    \end{pmatrix}
    .
\]

The theorems from \link{http://www.math.unl.edu/~sdunbar1/ProbabilityTheory/Lessons/MarkovChains/WaitingTimeAbsorption/waitingtimeabsorption.pdf}
{Waiting Time to Absorption} give a lot of information about the
progress of the game.  The entry \( N_ {ij} \) (counting \( i \) and \(
j \) from \( 0 \)) represents the expected number of turns spent in
state \( j \) given that the player started in state \( i \).  Notice
that the diagonal entries are all greater than \( 1 \).  If a player
starts the game in state \( i \) then at least one roll, the first roll,
is in state \( i \).  Entries in later states \( j>i \) can be less than
\( 1 \) because it is possible for a player to skip a state entirely.
Notice the \( 6 \) in the bottom right corner gives the average number
of rolls spent in state \( 9 \) given that the player started in state \(
9 \).  This makes sense; in state \( 9 \), the player has just one die
left to roll.  It will take \( 6 \) rolls on average to roll the given
target number, this is the mean of a geometric random variable with \( p
= 1/6 \).  Since players start the game of \emph{Tenzi} in state \( 0 \)
the first row of matrix \( N \) gives the expected number of rolls a
player will spend in each state, see Table~%
\ref{tab:games:tenzimean}.
\begin{table}
    \centering
    \begin{tabular}{rrrrrrrrrr}
        0    & 1    & 2    & 3    & 4    & 5    & 6    & 7    & 8    & 9    \\ 
        1.19 & 0.48 & 0.67 & 0.79 & 0.92 & 1.10 & 1.37 & 1.83 & 2.74 & 5.48
    \end{tabular}
    \caption{Expected number of times the chain will be in each state.}%
    \label{tab:games:tenzimean}
\end{table}
This makes sense, for state \( 0 \), there is always the first roll, but
usually the player moves to another state.  For states \( 1 \) through \(
4 \), it is common to get more than one of the target number, so those
states often get skipped over.  For states \( 5 \) to \( 9 \), it takes
increasingly longer to collect the target number.

The sum of the first row is the mean number of turns to go from state \(
0 \) to the absorbing state \( 10 \), specifically, \( 16.564849 \),
agreeing with the approximate mean from the pdf.  The standard deviation
of the number of turns the player will spend in state \( i \) before
reaching the absorbing state \( 10 \) from state \( 0 \) is in Table~%
\ref{tab:games:tenzisd}.
\begin{table}
    \centering
    \begin{tabular}{rrrrrrrrrr}
        0    & 1    & 2    & 3    & 4    & 5    & 6    & 7    & 8    & 9    \\ 
        6.83 & 6.81 & 6.78 & 6.75 & 6.70 & 6.64 & 6.55 & 6.41 & 6.14 & 5.48
    \end{tabular}
    \caption{Standard deviation of the number of times the chain will be
    in each state.}%
    \label{tab:games:tenzisd}
\end{table}
This also makes sense; in state \( 9 \), then a player has just one die
left to roll.  The entry \( 5.48 \) is the approximate standard
deviation of a geometric random variable with \( p = 1/6 \).

\subsection*{Random Moves in the Towers of Hanoi}

The Towers of Hanoi (also called the Tower of Brahma or Lucas' Tower) is
a mathematical game or puzzle.  It consists of three rods and some
number of disks of different sizes, with holes in the center so each
disk can slide onto any rod.  The puzzle starts with the disks in a neat
stack in ascending order of size on one rod, the smallest at the top,
making a conical shape.  The objective of the puzzle is to move the
entire stack to another rod, obeying \( 3 \) rules.
\begin{enumerate}
    \item
        Only one disk can be moved at a time.
    \item
        Each move consists of taking the top disk from one of the stacks
        and placing it on top of another stack or on an empty rod.
    \item
        No larger disk may be placed on top of a smaller disk.
\end{enumerate}

Consider solving the small three disk case of Lucas' Tower puzzle with
random legal moves.  The disks start with the biggest disk at the
bottom, the intermediate size disk in the middle, and the smallest disk
on top.  The goal is to move all three disks, one at a time, from one
pole to another pole.  At no point can a larger disk sit atop a smaller
disk.  See Figure~%
\ref{fig:games:towerhanoi}
\begin{figure}
    \centering
\begin{asy}
size(5inches);

real myfontsize = 12;
real mylineskip = 1.2*myfontsize;
pen mypen = fontsize(myfontsize, mylineskip);
defaultpen(mypen);

real basew = 3.0;
real baseh = 0.10;

real rodw = 0.1;
real rodh = 0.75;

real disk1diam  = 0.25;
real disk2diam  = 0.50;
real disk3diam  = 0.75;
real diskh      = 0.1;
real diskspace  = 0.05;

real labelspace = 0.1;

pair base1 = (0,0 );
pair base2 = (basew,baseh);

pair roda1 = (basew/4 - rodw/2, baseh);
pair roda2 = (basew/4 + rodw/2, baseh+rodh);

pair rodb1 = (2*basew/4 - rodw/2, baseh);
pair rodb2 = (2*basew/4 + rodw/2, baseh+rodh);

pair rodc1 = (3*basew/4 - rodw/2, baseh);
pair rodc2 = (3*basew/4 + rodw/2, baseh+rodh);

pair disk31 = (basew/4 - disk3diam/2, baseh + diskspace);
pair disk32 = (basew/4 + disk3diam/2, baseh + diskspace + diskh);

pair disk21 = (basew/4 - disk2diam/2, baseh + 2*diskspace + diskh);
pair disk22 = (basew/4 + disk2diam/2, baseh + 2*diskspace + 2*diskh);

pair disk11 = (basew/4 - disk1diam/2, baseh + 3*diskspace + 2*diskh);
pair disk12 = (basew/4 + disk1diam/2, baseh + 3*diskspace + 3*diskh);

picture p = new picture;
size(p, 2inches);

filldraw(p, box(base1, base2) );

filldraw(p, box(roda1, roda2) );
filldraw(p, box(rodb1, rodb2) );
filldraw(p, box(rodc1, rodc2));

pair disk31 = (basew/4 - disk3diam/2, baseh + diskspace);
pair disk32 = (basew/4 + disk3diam/2, baseh + diskspace + diskh);

pair disk21 = (basew/4 - disk2diam/2, baseh + 2*diskspace + diskh);
pair disk22 = (basew/4 + disk2diam/2, baseh + 2*diskspace + 2*diskh);

pair disk11 = (basew/4 - disk1diam/2, baseh + 3*diskspace + 2*diskh);
pair disk12 = (basew/4 + disk1diam/2, baseh + 3*diskspace + 3*diskh);

filldraw(p, box(disk31, disk32), gray );
filldraw(p, box(disk21, disk22), gray );
filldraw(p, box(disk11, disk12), gray );

label( p, "$a$", (  basew/4, baseh + rodh + labelspace) );
label( p, "$b$", (2*basew/4, baseh + rodh + labelspace) );
label( p, "$c$", (3*basew/4, baseh + rodh + labelspace) );

picture q = new picture;
size(q, 2inches);

filldraw(q, box(base1, base2) );

filldraw(q, box(roda1, roda2) );
filldraw(q, box(rodb1, rodb2) );
filldraw(q, box(rodc1, rodc2) );

pair disk31 = (2*basew/4 - disk3diam/2, baseh + diskspace);
pair disk32 = (2*basew/4 + disk3diam/2, baseh + diskspace + diskh);

pair disk21 = (3*basew/4 - disk2diam/2, baseh + diskspace);
pair disk22 = (3*basew/4 + disk2diam/2, baseh + diskspace + diskh);

pair disk11 = (3*basew/4 - disk1diam/2, baseh + diskspace + 2*diskh);
pair disk12 = (3*basew/4 + disk1diam/2, baseh + diskspace + 3*diskh);


filldraw(q, box(disk31, disk32), gray );
filldraw(q, box(disk21, disk22), gray );
filldraw(q, box(disk11, disk12), gray );

label( q, "$a$", (  basew/4, baseh + rodh + labelspace) );
label( q, "$b$", (2*basew/4, baseh + rodh + labelspace) );
label( q, "$c$", (3*basew/4, baseh + rodh + labelspace) );

add(p.fit(),(0,0), (0,0) );
add(q.fit(),(0,0), (100,0) );
\end{asy}
    \caption{Starting configuration and a typical state in the Towers of
    Hanoi puzzle.}%
    \label{fig:games:towerhanoi}
\end{figure}

The minimum number of moves to solve the puzzle is well known:  \( m \)
disks can be solved in exactly \( 2^m-1 \) moves.  The maximum number of
legal moves to move all three disks from one pole to any other pole
without repeating a configuration is \( 3^m-1 \).  The question here is:
on average, how many moves does it take to solve this puzzle with \emph{random}
legal moves of three disks?

Label the rods from left to right as \( a \), \( b \), \( c \) as in
Figure~%
\ref{fig:games:towerhanoi}.  Represent the states for \( 3 \) disks as a
triplet of values from \( \set{a,b,c} \).  The first entry in the
triplet gets the rod label on which the smallest disk rests, the second
entry in the triplet gets the rod label for the middle disk, and the
third entry in the triplet gets the rod label on which the largest disk
rests.  So the start is \( aaa \), the goal is either \( bbb \) or \(
ccc \).  The configuration \( aba \) implies the smallest disk is on top
of the largest disk, while the medium disk is on the middle rod labeled \(
b \).  The puzzle with \( 3 \) disks has \( 27 \) legal configurations.

The triplet labeling leads to an example of a \emph{puzzle graph}, in
which the configurations of a puzzle or game are the nodes of a graph
and legal moves between states are the edges of the puzzle graph.
Figure~%
\ref{fig:games:puzzlegraph} is the puzzle graph for the Towers of Hanoi.
For example, the edge connecting \( aaa \) with \( baa \) represents the
initial move of the smallest disk to the middle rod.  The puzzle graph
for the Towers of Hanoi has a recursive ternary structure useful for
analyzing the shortest and longest paths from the start \( aaa \) to the
goals \( bbb \) or \( ccc \).  For this small random game just number
the states from the source, top to bottom, left to right, as in Figure~%
\ref{fig:games:puzzlegraph}

\begin{figure}
    \centering
\begin{asy}
size(5inches);

real myfontsize = 12;
real mylineskip = 1.2*myfontsize;
pen mypen = fontsize(myfontsize, mylineskip);
defaultpen(mypen);

//////////////////////////////////////////////////////
// code for the 'circle' shape from Nguyen Van Chi, 
// inspired by the 'roundbox' shape (see 'plain_boxes.asy')
path circle(frame dest, frame src=dest, real xmargin=0, real ymargin=xmargin,
            pen p=currentpen, filltype filltype=NoFill, bool above=true)
{ 
pair m=min(src);
pair M=max(src);
pair bound=M-m;
real a=bound.x+2*xmargin;
real b=bound.y+2*ymargin;
path g=shift(m-(xmargin,ymargin))*(circle((a/2,b/2),max(a,b)/2));
frame F;
 if(above == false) {
    filltype.fill(F,g,p);
    prepend(dest,F);
  } else filltype.fill(dest,g,p);
return g;
}

path circle(frame f, Label L, real xmargin=0, real ymargin=xmargin,
            pen p=currentpen, filltype filltype=NoFill, bool above=true)
{
add(f,L);
return circle(f,xmargin,ymargin,p,filltype,above);
}
//////////////////////////////////////////////////////

void Puzzlegraph(pair A, real s, int q, bool top=true)
{
  pair B=A-(1,sqrt(2))*s/2;
  pair C=B+s;
  if(top) {
    draw(A--B--C--cycle);
  }
  if(q > 0) {
    Puzzlegraph(A,s/3,q-1,true);
    Puzzlegraph( (1/3)*A+(2/3)*B, s/3,q-1,true);
    Puzzlegraph( (1/3)*A+(2/3)*C, s/3,q-1,true);
  }
}
//////////////////////////////////////////////////////


real s = 1;
pair A = (0,1);
pair B=A-(1,sqrt(2))*s/2;
pair C=B+s;


Puzzlegraph((0,1),1,2);

real margin=1.5mm;

object aaa = draw("$aaa$", circle, A, FillDraw(white, black));
object bbb = draw("$bbb$", circle, B, FillDraw(white, black));
object ccc = draw("$ccc$", circle, C, FillDraw(white, black));

object baa = draw("$baa$", circle, (8/9)*A + (1/9)*B, FillDraw(white, black));
object bca = draw("$bca$", circle, (7/9)*A + (2/9)*B, FillDraw(white, black));
object cca = draw("$cca$", circle, (6/9)*A + (3/9)*B, FillDraw(white, black));
object ccb = draw("$ccb$", circle, (1/3)*A + (2/3)*B, FillDraw(white, black));
object acb = draw("$acb$", circle, (2/9)*A + (7/9)*B, FillDraw(white, black));
object abb = draw("$abb$", circle, (1/9)*A + (8/9)*B, FillDraw(white, black));

object caa = draw("$caa$", circle, (8/9)*A + (1/9)*C, FillDraw(white, black));
object cba = draw("$cba$", circle, (7/9)*A + (2/9)*C, FillDraw(white, black));
object bba = draw("$bba$", circle, (6/9)*A + (3/9)*C, FillDraw(white, black));
object bbc = draw("$bbc$", circle, (3/9)*A + (6/9)*C, FillDraw(white, black));
object abc = draw("$abc$", circle, (2/9)*A + (7/9)*C, FillDraw(white, black));
object acc = draw("$acc$", circle, (1/9)*A + (8/9)*C, FillDraw(white, black));

object cbb = draw("$cbb$", circle, (8/9)*B + (1/9)*C, FillDraw(white, black));
object cab = draw("$cab$", circle, (7/9)*B + (2/9)*C, FillDraw(white, black));
object aab = draw("$aab$", circle, (6/9)*B + (3/9)*C, FillDraw(white, black));
object aac = draw("$aac$", circle, (3/9)*B + (6/9)*C, FillDraw(white, black));
object bac = draw("$bac$", circle, (2/9)*B + (7/9)*C, FillDraw(white, black));
object bcc = draw("$bcc$", circle, (1/9)*B + (8/9)*C, FillDraw(white, black));

object aca = draw("$aca$", circle, (6/9)*A + (2/9)*B + (1/9)*C, FillDraw(white, black));
object aba = draw("$aba$", circle, (6/9)*A + (1/9)*B + (2/9)*C, FillDraw(white, black));

object bcb = draw("$bcb$", circle, (2/9)*A + (6/9)*B + (1/9)*C, FillDraw(white, black));
object bab = draw("$bab$", circle, (1/9)*A + (6/9)*B + (2/9)*C, FillDraw(white, black));

object cbc = draw("$cbc$", circle, (2/9)*A + (1/9)*B + (6/9)*C, FillDraw(white, black));
object cac = draw("$cac$", circle, (1/9)*A + (2/9)*B + (6/9)*C, FillDraw(white, black));
\end{asy}
    \caption{The puzzle graph for the three disk Towers of Hanoi.}%
    \label{fig:games:puzzlegraph}
\end{figure}

Selecting each move uniformly randomly from the set of all valid moves
from any node to any adjacent node (except for the goal nodes which are
absorbing) defines a random walk on the graph or a Markov chain.  The
two goals, now states \( 20 \) and \( 27 \), are absorbing states.  Most
transition probabilities between states are \( 1/3 \), except for the
start and end states.  The recursive ternary labeling of the nodes is
not immediately useful for analysis of this \( 27 \)-state Markov chain.
The script for this Markov chain calculates the mean absorption time as \(
70.77 \) moves.  T. Berger and M. Alekseyev
\cite{Alekseyev2015} give two proofs that the mean time to absorption
for the \( N \)-disk Tower of Hanoi puzzle with random moves is
\[
    \frac{(3^m-1)(5^m- 3^m)}{4 \cdot 3^{m-1}}.
\] One proof uses recurrence relations, the other proof uses an
equivalence of Markov chains to electrical networks.

\subsection*{Chutes and Ladders}

The childrens' game \emph{Chutes and Ladders} (known in the United
Kingdom as \emph{Snakes and Ladders}) derives from an old Hindu game
\emph{Leela} or \emph{Moksha Patam} teaching moral lessons.  The game
described here is the modified U.S.\ version.  The game is a \( 100 \)
state absorbing Markov chain and the goal is to investigate the expected
playing time.~%
\index{\emph{Chutes and Ladders}}

The game is for two or more players.  The game play is on a \( 10 \) by \(
10 \) board with squares numbered \( 1 \) to \( 100 \) that are the
states of the game.  Rather than add an additional state \( 0 \) as the
starting position, instead make an initial uniform distribution of
starting states, see below for details.  (The scripts in the R
statistical computing language which indexes vectors from \( 1 \) in
contrast to say, Python, which indexes from \( 0 \) motivates this
choice of \( 100 \) states with an initial distribution.  With \( 101 \)
states including a starting state, the state indices would be one
greater than the spaces marked on the board, leading to potential
confusion.) Since the goal is to find the expected playing time, add \(
1 \) to the playing time for the initial move to be in one of the
starting states.  Players roll a fair six-sided die (or use a fair
six-section spinner) to determine how many squares to advance their
token on each turn.  Two or more players may occupy the same square
simultaneously.  The first player to reach square \( 100 \) exactly
wins.  In the final \( 5 \) states, if a move would take a player beyond
the last square, the player does not move.  The game board is also
marked with chutes and ladders connecting various squares on the board.
The special feature of the game is that if a token lands on a square at
the bottom of a ladder the token immediately moves to the top of the
ladder.  Similarly if the token lands on a square at the top of a chute,
it immediately moves to the bottom of the chute.  The game has \( 10 \)
chutes and \( 9 \) ladders with beginning and end states given by the
following ordered pairs
\begin{itemize}
    \item
        Ladders:  \( (1,38) \), \( (4,14) \), \( (9,31) \), \( (21,42) \),
        \( (28,84) \), \( (36, 44) \), \( (51,67) \), \( (71,91) \), \(
        (80, 100) \);
    \item
        Chutes:  \( (16,6) \), \( (47,26) \), \( (49,11) \), \( (56,53) \),
        \( (62,19) \), \( (64,60) \), \( (87,24) \), \( (93,73) \), \( (95,75)
        \), \( (98,78) \).
\end{itemize}
This means that \( 19 \) states, the beginnings of the chutes and
ladders, are inaccessible states.  For example, a player is never
actually in state \( 1 \), for if the token lands on square \( 1 \), the
ladder \( (1,38) \) immediately deposits the player into state \( 38 \).
Although the game is a race among two or more players, the players move
independently, so it is sufficient to consider just one player moving
and ask ``How long does it typically take a single player to finish a
game by reaching square \( 100 \)?''

Although the game has \( 100 \) squares, it is possible to model the
game as a Markov chain with fewer states, see
\cite{althoen93}.  Adding a starting state \( 0 \), but deleting each
inaccessible state corresponding to a chute or a ladder leaves \( 82 \)
states.  The probability transition matrix is an \( 82 \times 82 \)
matrix with most rows containing \( 6 \) consecutive probabilities of \(
1/6 \) each.  There are two kinds of exceptions.  A row corresponding to
any state reachable from a chute or ladder must account for the
long-distance transitions by the chute or ladder.  The last \( 5 \) rows
show the fact that a player must reach state \( 100 \) exactly, a die
roll causing the player to go past state \( 100 \) results in no move.

Here the model uses a \( 100 \)-state Markov chain with a \( 100 \times
100 \) transition probability matrix derived from matrices analogous to
those in the Monopoly model.  First use a \( 100 \times 100 \) circulant
(almost, excepting the last \( 5 \) rows) dice roll matrix \( R \)
followed by sparse \( 100 \times 100 \) chutes and ladders transition
matrices \( C \) and \( L \).  The matrix \( L \) is just the identity
matrix but, for instance the \( 1 \) in row \( 1 \) and column \( 1 \),
moves to row \( 1 \) and column \( 38 \) to indicate ladder \( (1,38) \).
Likewise for the \( 18 \) other ladders and chutes.  The result \( RLC \)
is a \( 100 \times 100 \) transition probability matrix with transient
states \( 1 \) through \( 99 \) and absorbing state \( 100 \).  The game
starts in states \( 38 \), \( 2 \), \( 3 \), \( 14 \), \( 5 \) and \( 6 \),
each with probability \( \frac{1} {6} \).  The expected waiting time to
absorption is the average over the expected waiting time to absorption
from each state, \( \frac{1}{6}\left( W(38) + W(2) + W(3) + W(14) + W(5)
+ W(6) \right) = 38.22512 \).  Adding \( 1 \) for the initial move to
the starting states gives \( 39.22512 \), agreeing with
\cite{althoen93}.  Using Theorem 3 from \link{http://www.math.unl.edu/~sdunbar1/ProbabilityTheory/Lessons/MarkovChains/WaitingTimeAbsorption/waitingtimeabsorption.pdf}
{Waiting Time to Absorption} the standard deviation of the waiting time
to absorption is approximately \( 25.2 \) turns.  As a final note,
observe that the waiting time to absorption from state \( 99 \) is \( 6 \)
with standard deviation \( \sqrt{30} \approx 5.48 \), exactly as
expected for a geometric random variable with \( p=1/6 \), waiting to
roll a \( 1 \) to enter the last state.~%
\index{geometric random variable}

In
\cite{althoen93} the authors investigate the sensitivity of the waiting
time to adding or removing chutes and ladders.  For example, the
expected waiting time to absorption for a game with no chutes and
ladders is \( 34.68571 \) moves.  See the exercises for some examples.

\subsection*{Count Your Chickens}

\emph{Count Your Chickens!}, produced by Peaceable Kingdom, is a simple
game for preschoolers.~%
\index{\emph{Count Your Chickens!}}
In this cooperative game, play begins with forty chicks spread
throughout the board and a mama chicken at the start square.  The end
square is the mama chicken's coop, and the mama chicken moves
progressively closer to this square as the game unfolds.  The object of
the game is to have all \( 40 \) chicks in the chicken coop when the
mama chicken arrives there.  The game board has \( 41 \) squares
numbered in order, with \( 30 \) squares containing one of five
characters:  a cow, a sheep, a pig, a dog, or a tractor.  The remaining \(
11 \) squares are empty and neutral.  Five of the character squares have
a blue background.  The final square, the chicken coop, contains all \(
5 \) farm characters.  Players take turns spinning a spinner with 6
segments of equal area.  Five segments match the farm characters from
the board and a sixth segment has a fox character.  When a player spins
the spinner, one of two things can happen.  If the player spins one of
the five characters drawn on the board, she moves the mama chicken token
along the path toward the coop to the next instance of the character.
The player counts the squares moved, and adds this number of chicks to
the chicken coop.  Additionally, if the square landed on has a blue
background, the player adds one additional chick to the coop.  There are
only a total of \( 40 \) chicks on the board though, so if the mama
chicken moves two squares but there are already 39 chicks in the coop,
only one chick gets added.  If a fox is spun, the player must remove one
chick from the coop and place it back on the board.  If there are no
chicks in the coop then spinning a fox has no effect.

For example, suppose the mama chicken is at the start square.  If the
player spins a sheep, then the mama chicken moves to square \( 2 \) (the
first sheep) and places two chicks in the coop.  If the player next
spins a tractor then the mama chicken moves \( 2 \) squares to square \(
4 \) and places \( 3 \) more chicks in the coop because square \( 4 \)
is blue.  If the first spin is a fox then nothing happens because at the
start of the game there are no chicks in the coop.  The game continues
in this manner until the mama chicken reaches the coop.  The players win
if all \( 40 \) chicks are in the coop when the mama chicken arrives
there; otherwise the players lose, this is a cooperative game.

The goal here is to answer the following questions:
\begin{enumerate}
    \item
        What is the probability of winning a game of \emph{Count Your
        Chickens!}?  That is, what is the probability of landing on
        square \( 40 \) with \( 40 \) chicks in the coop?
    \item
        What is the expected number of chicks in the coop at the end of
        the game?
    \item
        What effect do the blue squares have on the probability of
        winning the game?  For example, the original game has five
        particular squares colored blue; are there other choices of five
        blue squares that increase the probability of winning the game?
\end{enumerate}

The mathematical interest here is creating the state space, leading to a
large transition probability matrix.  A state in \emph{Count Your
Chickens!} is an ordered pair \( (a, b) \) where \( a \) is the mama
chicken's position on the board and \( b \) is the number of chicks in
the coop.  For win probabilities only, to minimize the number of states
and cut down on computation, include a ``loss'' state and a ``win''
state.  The idea is that while \( (39, 35) \), for example, is a
possible state of the game, it is impossible to win from such a
configuration.  Lump together all such ordered pairs from which it is
impossible to win as a loss state.  The win state is \( (40, 40) \), the
mama chicken must reach the last square of the board with all of the
chicks in the coop.

The following example illustrates the construction of such a chain with
a smaller board, see Figure~%
\ref{fig:games:smallchickens}.

\begin{example}
    This smaller game uses only two farm characters -- a sheep and a
    cow.  There are \( 8 \) squares on the board after the start square
    and \( 8 \) chicks to return to the coop.  The spinner has three
    equal-sized segments illustrated with a cow, a sheep, and a fox.
    Number only the squares with a character, since it is not possible
    to land on a blank square.

    \begin{figure}
        \centering
\begin{asy}
        size(5inches);

        real myfontsize = 12;
        real mylineskip = 1.2*myfontsize;
        pen mypen = fontsize(myfontsize, mylineskip);
        defaultpen(mypen);

        import math;

        for (int j=0; j<9; ++j){ label(format("%i", j), (j,1));}

          add( shift(-0.2, 0.2)*grid(9, 1) );
          fill( box((1.8,0.2), (2.8,1.2)), lightblue+opacity(0.5));
          fill( box((4.8,0.2), (5.8, 1.2)),lightblue+opacity(0.5));

          label("Start", (0.3, 0.7));
          label("Sheep", (2.3, 0.7));
          label("Cow",   (3.3, 0.7));
          label("Cow",   (5.3, 0.7));
          label("Sheep", (7.3, 0.7));

          label("\footnotesize{Coop}", (8.45, 1.0));
          label("Sheep", (8.3, 0.7));
          label("Cow",   (8.3, 0.4));
\end{asy}
        \caption{Smaller game board for the example \emph{Count Your
        Chickens!} }%
        \label{fig:games:smallchickens}
    \end{figure}

    A state of the game is an ordered pair \( (i, j) \) where \( i \) is
    the position of the mama chicken and \( j \) is the number of chicks
    in the coop.  Identify some pairs with the loss state because if \(
    j \) is small enough relative to the mama chicken's position, it
    will be impossible for all \( 8 \) chicks to be in the coop at the
    end of the game.  The state \( (2, 0) \) is such an example:  if the
    mama chicken is on square \( 2 \) and there are no chicks in the
    coop, the game will end with at most \( 7 \) chicks in the coop.
    Let \( b_i^R \) be the number of blue squares after square \( i \)
    on the path and \( b_i^L \) the number of blue squares that come
    before square \( i \), including square \( i \).  Note \( b_i^R + b_i^L
    \) is the total number of blue squares in the game.  Then for square
    \( i \), the states are \( (i, j ) \) with \( \max(i - b_i^R, 0) \le
    j \le \min(i + b_i^L, 8) \).  For example, for square \( 3 \), \( b_3^R
    = 1 \) and \( b_3^L = 1 \) since there is one blue square to the
    right of square \( 3 \) and one blue square to the left of square \(
    3 \).  Thus square \( 3 \) corresponds to possible states \( (3, 2) \),
    \( (3, 3) \), \( (3, 4) \).  It is not possible to have more than \(
    4 \) chicks in the coop on square \( 3 \), and if the coop has \( 0 \)
    or \( 1 \) chicks on square \( 3 \) it is impossible to win the
    game.  All the states are
    \begin{multline*}
        \left\{(0, 0), (2, 1), (2, 2), (2, 3), (3, 2), (3, 3), \right.\\
        \left.  (3, 4), (5,5), (5, 6), (5, 7), (7, 7), (7, 8), \text{win},
        \text{loss} \right\}
    \end{multline*}
    for this board.  For the purposes of building a transition matrix
    think of \( (0, 0) \) as state \( 1 \), \( (2, 1) \) as state \( 2 \),
    \( (2, 2) \) as state \( 3 \), \dots, and ``loss'' as state \( 14 \).
    Entry \( p_{i,j} \) in the transition matrix for the game is the
    probability of moving from state \( i \) to state \( j \).  For
    example, in row \( 1 \) of the transition matrix, start in position \(
    (0, 0) \) on the game board.  There is a \( 1/3 \) probability of
    rolling a fox; this spin keeps the token in position \( (0, 0) \),
    so \( p_{1,1} = 1/3 \).  Spinning a sheep is the next possibility,
    putting the token in position \( (2, 3) \), or a cow, putting the
    token in position \( (3, 3) \).  This gives \( p_{1,4} = 1/3 \) and \(
    p_{1,6} = 1/3 \).  All other entries in row \( 1 \) are \( 0 \).
    Starting in state \( (2, 1) \), then spinning the sheep puts the
    token in position \( (7, 6) \), or the loss state, because if there
    are only \( 6 \) chicks in the coop upon reaching square \( 7 \), it
    is impossible to win.  Spinning the fox also puts the game in the
    loss state, so entry \( p_{2,14} = 2/3 \).  Spinning the cow puts
    the game in state \( (3, 2) \), so \( p_{2,5} = 1/3 \).  All other
    entries in row two are \( 0 \).  Continuing in this way, the
    transition matrix for this game is
    \[
        \begin{pmatrix}
            \frac{1}{3} & 0     & 0     & \frac{1}{3}   & 0     & \frac{1}
            {3} & 0     & 0     & 0     & 0     & 0     & 0     & 0
            & 0\\
            0   & 0     & 0     & 0     & \frac{1}{3}   & 0     & 0
            & 0 & 0     & 0     & 0     & 0     & 0     & \frac{2}{3}\\
            0   & \frac{1}{3}   & 0     & 0     & 0     & \frac{1}{3}
            & 0 & 0     & 0     & 0     & \frac{1}{3}   & 0     & 0
            & 0\\
            0   & 0     & \frac{1}{3}   & 0     & 0     & 0     & \frac{1}
            {3} & 0     & 0     & 0     & 0     & \frac{1}{3}   & 0
            & 0\\
            0   & 0     & 0     & 0     & 0     & 0     & 0     & \frac{1}
            {3} & 0     & 0     & 0     & 0     & 0     & \frac{2}{3}\\
            0   & 0     & 0     & 0     & \frac{1}{3}   & 0     & 0
            & 0 & \frac{1}{3}   & 0     & \frac{1}{3}   & 0     & 0
            & 0\\
            0   & 0     & 0     & 0     & 0     & \frac{1}{3}   & 0
            & 0 & 0     & \frac{1}{3}   & 0     & \frac{1}{3}   & 0
            & 0\\
            0   & 0     & 0     & 0     & 0     & 0     & 0     & 0
            & 0 & 0     & \frac{1}{3}   & 0     & \frac{1}{3}   & \frac{1}
            {3}\\
            0   & 0     & 0     & 0     & 0     & 0     & 0     & \frac{1}
            {3} & 0     & 0     & 0     & \frac{1}{3}   & \frac{1}{3}
            & 0\\
            0   & 0     & 0     & 0     & 0     & 0     & 0     & 0
            & \frac{1}{3}       & 0     & 0     & \frac{1}{3}   & \frac{1}
            {3} & 0\\
            0   & 0     & 0     & 0     & 0     & 0     & 0     & 0
            & 0 & 0     & 0     & 0     & \frac{2}{3}   & \frac{1}{3}\\
            0   & 0     & 0     & 0     & 0     & 0     & 0     & 0
            & 0 & 0     & \frac{1}{3}   & 0     & \frac{2}{3}   & 0\\
            0   & 0     & 0     & 0     & 0     & 0     & 0     & 0
            & 0 & 0     & 0     & 0     & 1     & 0\\
            0   & 0     & 0     & 0     & 0     & 0     & 0     & 0
            & 0 & 0     & 0     & 0     & 0     & 1
        \end{pmatrix}
        .
    \]

    The state and transition probability matrix models the small game as
    an absorbing Markov chain.  The loss and win states are both
    absorbing since each of these states transitions to itself with
    probability \( 1 \), and it is possible to reach one of these two
    states from any other state.  The transition matrix is in the
    canonical form
    \[
        \begin{pmatrix}
            Q   & R \\
            0   & I
        \end{pmatrix}
    \] where \( I \) is the identity matrix.  With the transition matrix
    in canonical form, the fundamental matrix is
    \[
        N = (I - Q)^{-1} =
        \begin{pmatrix}
            \frac{3}{2} & \frac{1}{18}  & \frac{1}{6}   & \frac{1}{2}
            & \frac{2}{9}       & \frac{11}{18} & \frac{1}{6}   & \frac{4}
            {27}        & \frac{2}{9}   & \frac{1}{18}  & \frac{67}{162}
                & \frac{17}{54}\\
            0   & 1     & 0     & 0     & \frac{1}{3}   & 0     & 0
            & \frac{1}{9}       & 0     & 0     & \frac{1}{27}  & 0\\
            0   & \frac{1}{3}   & 1     & 0     & \frac{2}{9}   & \frac{1}
            {3} & 0     & \frac{1}{9}   & \frac{1}{9}   & 0     & \frac{40}
            {81}        & \frac{1}{27}\\
            0   & \frac{1}{9}   & \frac{1}{3}   & 1     & \frac{1}{9}
            & \frac{2}{9}       & \frac{1}{3}   & \frac{2}{27}  & \frac{1}
            {9} & \frac{1}{9}   & \frac{31}{81} & \frac{14}{27}\\
            0   & 0     & 0     & 0     & 1     & 0     & 0     & \frac{1}
            {3} & 0     & 0     & \frac{1}{9}   & 0\\
            0   & 0     & 0     & 0     & \frac{1}{3}   & 1     & 0
            & \frac{2}{9}       & \frac{1}{3}   & 0     & \frac{4}{9}
            & \frac{1}{9}\\
            0   & 0     & 0     & 0     & \frac{1}{9}   & \frac{1}{3}
            & 1 & \frac{1}{9}   & \frac{2}{9}   & \frac{1}{3}   & \frac{26}
            {81}        & \frac{14}{27}\\
            0   & 0     & 0     & 0     & 0     & 0     & 0     & 1
            & 0 & 0     & \frac{1}{3}   & 0\\
            0   & 0     & 0     & 0     & 0     & 0     & 0     & \frac{1}
            {3} & 1     & 0     & \frac{2}{9}   & \frac{1}{3}\\
            0   & 0     & 0     & 0     & 0     & 0     & 0     & \frac{1}
            {9} & \frac{1}{3}   & 1     & \frac{5}{27}  & \frac{4}{9}\\
            0   & 0     & 0     & 0     & 0     & 0     & 0     & 0
            & 0 & 0     & 1     & 0\\
            0   & 0     & 0     & 0     & 0     & 0     & 0     & 0
            & 0 & 0     & \frac{1}{3}   & 1
        \end{pmatrix}
        .
    \]%
    \index{fundamental matrix}
    Let \( B = N R \).  If the game is in the \( i \)th non-absorbing
    state then the probability that the game terminates by reaching the \(
    j \)th absorbing state is \( b_{ij} \).  Then
    \[
        B =
        \begin{pmatrix}
            \frac{305}{486}     & \frac{181}{486}\\
            \frac{5}{81}        & \frac{76}{81}\\
            \frac{104}{243}     & \frac{139}{243}\\
            \frac{170}{243}     & \frac{73}{243}\\
            \frac{5}{27}        & \frac{22}{27}\\
            \frac{5}{9} & \frac{4}{9}\\
            \frac{190}{243}     & \frac{53}{243}\\
            \frac{5}{9} & \frac{4}{9}\\
            \frac{22}{27}       & \frac{5}{27}\\
            \frac{73}{81}       & \frac{8}{81}\\
            \frac{2}{3} & \frac{1}{3}\\
            \frac{8}{9} & \frac{1}{9}
        \end{pmatrix}
        .
    \]
\end{example}
From the first row of \( B \), this game has a win probability of \(
305/486 \approx 0.6276 \) and a loss probability of \( 181/486 \approx
0.3724 \) when a player begins from the start square.  One of the nice
aspects of using a Markov chain for these calculations is that a player
can update the win probability moving through the game.  For example,
suppose in two turns the player starts by spinning a sheep and then
spins a fox.  The coop would first gain \( 3 \) chicks, then lose a
chick. This state of the game corresponds to the state \( (2, 2) \),
which corresponds to the third row of \( B \).  Therefore in such a
situation the player would know that her win probability is now only \(
104/243 \approx 0.4780 \).

The computations for the general game are similar except that the
matrices \( Q \) and \( R \) are much larger.  \( Q \) is \( 163 \times
163 \) and \( R \) is \( 163 \times 2 \) and therefore \( B \) is \( 163
\times 2 \).  It is impractical to enter and compute with such large
matrices by hand.  The Scripts section has an R program to build the
matrices and compute the win probability.  The matrix \( Q \) in the
full game looks similar to the matrix \( Q \) in the small example:  the
matrix is mostly zeroes punctuated sparsely with small diagonals of
entries of value \( 1/6 \), instead of \( 1/3 \).  The top left entry of
\( B \) in this case is \( 0.6410 \), and the probability of winning a
game of \emph{Count Your Chickens!} is \( 0.641 \).  The length of the
game from the start state to absorption in the win or loss states is \(
10.83 \) moves, with a varaince of \( 4.43 \) moves.

\visual{Section Starter Question}{../../../../CommonInformation/Lessons/question_mark.png}
\section*{Section Ending Answer} The future development of the game at a
turn must be independent of the past to model the game as a Markov
chain.  Games depending on a randomizing device, such as a dice rolls or
a spinner, can often be modeled as a Markov chain.  Examples in this
section are the games of \emph{Tenzi}, \emph{Chutes and Ladders}, and
\emph{Count Your Chickens!}.

\subsection*{Sources}

The subsection on \emph{Tenzi} is adapted from
\cite{bacinski20}. The \emph{Towers of Hanoi} example is adapted from
https://fivethirtyeight.com/features/can-you-cross-like-a-boss/
Background information is from
http://www.cut-the-knot.org/recurrence/hanoi.shtml Puzzle graphs are
examined in more detail in https://www.cut-the-knot.org/do\_you\_know/graphs2.shtml
The labeling of the states is from https://en.wikipedia.org/wiki/Tower\_of\_Hanoi
The reference to the Alekseyev-Berger paper is
\cite{Alekseyev2015} The subsection on Chutes and Ladders is adapted
from
\cite{althoen93}.  The subsection on Count Your Chickens is adapted
from
\cite{mccune19}.

\hr

\visual{Algorithms, Scripts, Simulations}{../../../../CommonInformation/Lessons/computer.png}
\section*{Algorithms, Scripts, Simulations}

\subsection*{Algorithm}
\begin{algorithm}[H]
  \DontPrintSemicolon
  \KwData{States and transition probabilities for Tenzi}
  \KwResult{Mean and standard deviation of waiting time to win}
  \KwResult{Plots of pdf and cdf of waiting time}
  \BlankLine
  \emph{Initialization of Matrices}\;
  Load transition probability matrix\;
  \For{$i = 2:50$}{
  Iteratively compute powers of P\;
  Add (1,11) entry to cdf list\;}
  Differences of cdf are the pdf\;
  Check total probability and approximate mean\;
  \BlankLine
  Extract transient probability submatrix\;
  Compute fundamental matrix with matrix inversion\;
  Calculate mean from fundamental matrix\;
  Calculate variance of waiting times\;
  Print mean and standard variation\;
  Plot pdf and cdf functions\;
  \caption{Tenzi Ladders Game Markov Chain.}
\end{algorithm}

\begin{algorithm}[H]
  \DontPrintSemicolon
  \KwData{States and transition probabilities for Chutes and Ladders}
  \KwResult{Mean and standard deviation of waiting time to win}
  \BlankLine
  \SetKwFunction{Frotvec}{rotvec}
  
  \SetKwProg{rot}{function}{}{}
  \rot{\Frotvec($v$)}{}{
    Cyclically rotate vector elements one place to right\;
    \Return rotated vector\;}

  \BlankLine
    
  \emph{Inititialization of Matrices}\;
  Load Markov Chain library\;
  Set number of states and state names\;
  Fill (almost) diceRoll matrix using \Frotvec\;
  Fill diagonal Ladder matrix with moves from ladders\;
  Fill diagonal Chutes matrix with moves from chutes\;
  Transition probability matrix as product of diceRoll, Ladders, Chutes
  Transition matrix is product of dice roll, ladders and chutes\;
  \emph{Markov chain calculations}\;
  Create Chutes and Ladders Markov chain object\;
  Compute mean absorption time with library function\;
  Extract transient probability submatrix\;
  Compute fundamental matrix with matrix inversion\;
  Calculate mean from fundamental matrix\;
  Calculate variance of waiting times\;
  Print mean and standard variation\;
  \caption{Chutes and Ladders Game Markov Chain.}
\end{algorithm}

\subsection*{Scripts}


\begin{description}

% \item[Geogebra] 

% \link{  .ggb}{GeoGebra applet}

\item[R] 

\link{http://www.math.unl.edu/~sdunbar1/MarkovChains/Games/tenzi.R}{R script for Tenzi.}

  \begin{lstlisting}[language=R]
x  <-  c(0.1615055828898457, 0.3230111657796915, 0.2907100492017223,
         0.1550453595742519, 0.05426587585098816, 0.01302381020423716,
         0.002170635034039527, 2.48072575318803*10^-4,
         1.860544314891023*10^-5, 8.2690858439601*10^-7,
         1.65381716879202*10^-8, 0.0, 0.1938066994678149,
         0.3488520590420667, 0.2790816472336534, 0.1302381020423716,
         0.03907143061271148, 0.007814286122542296,
         0.001041904816338973, 8.930612711476907*10^-5,
         4.465306355738455*10^-6, 9.922903012752121*10^-8, 0.0, 0.0,
         0.2325680393613778, 0.3721088629782046, 0.2604762040847432,
         0.1041904816338973, 0.02604762040847432, 0.004167619265355891,
         4.167619265355891*10^-4, 2.381496723060509*10^-5,
         5.953741807651273*10^-7,0.0, 0.0, 0.0, 0.2790816472336534,
         0.3907143061271148, 0.2344285836762689, 0.07814286122542295,
         0.01562857224508459, 0.001875428669410151,
         1.250285779606767*10^-4, 3.572245084590763*10^-6,0.0, 0.0,
         0.0, 0.0, 0.3348979766803841, 0.4018775720164609,
         0.2009387860082305, 0.05358367626886145, 0.008037551440329218,
         6.430041152263375*10^-4, 2.143347050754458*10^-5,0.0, 0.0,
         0.0, 0.0, 0.0, 0.4018775720164609, 0.4018775720164609,
         0.1607510288065844, 0.03215020576131687, 0.003215020576131687,
         1.286008230452675*10^-4,0.0, 0.0, 0.0, 0.0, 0.0, 0.0,
         0.4822530864197531, 0.3858024691358025, 0.1157407407407407,
         0.0154320987654321, 7.716049382716049*10^-4,0.0, 0.0, 0.0,
         0.0, 0.0, 0.0, 0.0, 0.5787037037037037, 0.3472222222222222,
         0.06944444444444445, 0.004629629629629629,0.0, 0.0, 0.0, 0.0,
         0.0, 0.0, 0.0, 0.0, 0.6944444444444444, 0.2777777777777778,
         0.02777777777777778,0.0, 0.0, 0.0, 0.0, 0.0, 0.0, 0.0, 0.0,
         0.0, 0.8333333333333334, 0.1666666666666667,0.0, 0.0, 0.0,
         0.0, 0.0, 0.0, 0.0, 0.0, 0.0, 0.0, 1.0)
P  <-  t( matrix(x, 11, 11) )

PPow  <- P
tenzicdf  <- c(PPow[1,11])
for (i in 2:50) {
  PPow  <- PPow %*% P;
  tenzicdf  <- c(tenzicdf, PPow[1,11])
}

tenzipdf  <-  tenzicdf[2:50] - tenzicdf[1:49]
tenzipdf  <- c(tenzicdf[1], tenzipdf)

sum(tenzipdf)
tenzimean  <- (1:50) %*% tenzipdf

Q  <-  P[1:10, 1:10]
I10  <-  diag(10)
N  <- solve( I10 - Q, I10)

mean  <- N %*% rep(1,10)
Varw  <- ((2 * N - I10) %*% N) %*% rep(1, 10) - ( N %*% rep(1, 10) )^2
sdw  <- sqrt(Varw)
cat("Mean waiting time: ", mean[1], "\nSD of waiting time:", sdw[1], "\n");

plot(x = 1:50, tenzicdf, col="red", xlab="Rolls", ylab="Probability")
points(1:50, tenzicdf, type="l", col="red")
points(1:50, tenzipdf, col="blue")
points(1:50, tenzipdf, type="l", col="blue")
\end{lstlisting}  

\item[R] 

\link{http://www.math.unl.edu/~sdunbar1/MarkovChains/Games/chutes.R}{R
  script for Chutes and Ladders.}

  \begin{lstlisting}[language=R]
library("markovchain")

rotvec <- function(vec) vec[ c(length(vec), 1:(length(vec)-1)) ]

nStates <- 100
chutesladdersStates <- as.character( c(1:nStates) )

diceRoll <- matrix(0, nStates, nStates)
p1 <- c(0,
        1/6, 1/6, 1/6, 1/6, 1/6, 1/6,
        numeric(nStates-7)
        )
for (i in 1:(nStates-5)) {
    diceRoll[i, ] <- p1
    p1 <- rotvec(p1)
}
diceRoll[100, 100] <- 1

diceRoll[99, 99] <- 5/6; diceRoll[99, 100] <- 1/6

diceRoll[98, 98] <- 4/6; diceRoll[98, 99] <- 1/6; diceRoll[98, 100] <- 1/6

diceRoll[97, 97] <- 3/6; diceRoll[97, 98] <- 1/6
diceRoll[97, 99] <- 1/6; diceRoll[97, 100] <- 1/6

diceRoll[96, 96] <- 2/6; diceRoll[96, 97] <- 1/6
diceRoll[96, 98] <- 1/6; diceRoll[96, 99] <- 1/6; diceRoll[96, 100] <- 1/6

ladders <- diag(nStates)
ladderIn <-  c( 1,  4,  9, 21, 28, 36, 51, 71,  80)
ladderOut <- c(38, 14, 31, 42, 84, 44, 67, 91, 100)
ladders[ cbind(ladderIn, ladderIn) ] <- 0
ladders[ cbind(ladderIn, ladderOut) ] <- 1

chutes <- diag(nStates)
chuteIn <-  c(16, 47, 49, 56, 62, 64, 87, 93, 95, 98)
chuteOut <- c( 6, 26, 11, 53, 19, 60, 24, 73, 75, 78)
chutes[ cbind(chuteIn, chuteIn) ] <- 0
chutes[ cbind(chuteIn, chuteOut) ] <- 1

chutesladdersMat <-  diceRoll %*% ladders %*% chutes

chutesladders <- new("markovchain", states = chutesladdersStates, byrow = TRUE,
                     transitionMatrix = chutesladdersMat, name = "ChutesLadders")
meanMC  <- meanAbsorptionTime(object=chutesladders)

Q  <- chutesladdersMat[1:99, 1:99]
I99  <-  diag(99)
N  <- solve( I99 - Q, I99)
absorptionTimeMat  <-  N %*% rep(1, 99)
meanMat  <- mean( absorptionTimeMat[ c(38, 2, 3, 14, 5, 6)] )
varabsorptionTimeMat  <- ((2 * N - I99) %*% N) %*% rep(1, 99) - ( N %*% rep(1, 99) )^2
sdAbsorptionTime  <-  sqrt( mean( varWait[ c(38, 2, 3, 14, 5, 6) ] ) )

cat("Mean waiting time from markovchain library: ", meanMC[1],
"\nMean waiting time from normal matrix: ", meanMat + 1,
"\nSD of waiting time:", sdAbsorptionTime[1], "\n");

\end{lstlisting}

\item[R] 

\link{http://www.math.unl.edu/~sdunbar1/MarkovChains/Games/countyourchickens.R}{R
  script for \emph{Count Your Chickens!}.}

\begin{lstlisting}[language=R]

boardLength <- 40

animalList <- c("C", "D", "P", "S", "T")
NAnimals <- length(animalList) + 1  # one more for fox

blueSquares <- numeric(boardLength)
blueSquares[4] <- 1
blueSquares[8] <- 1
blueSquares[22] <- 1
blueSquares[35] <- 1
blueSquares[39] <- 1

totalNumberOfBlues <- sum(blueSquares)

# Setting up the board
board <- character(boardLength)

board[1] <- "B"
board[2] <- "S"
board[3] <- "P"
board[4] <- "T"
board[5] <- "C"
board[6] <- "D"
board[7] <- "P"
board[8] <- "C"
board[9] <- "D"
board[10] <- "S"
board[11] <- "T"
board[12] <- "B"
board[13] <- "C"
board[14] <- "P"
board[15] <- "B"
board[16] <- "B"
board[17] <- "B"
board[18] <- "T"
board[19] <- "B"
board[20] <- "T"
board[21] <- "D"
board[22] <- "S"
board[23] <- "C"
board[24] <- "D"
board[25] <- "P"
board[26] <- "T"
board[27] <- "B"
board[28] <- "S"
board[29] <- "C"
board[30] <- "B"
board[31] <- "B"
board[32] <- "T"
board[33] <- "P"
board[34] <- "S"
board[35] <- "D"
board[36] <- "B"
board[37] <- "S"
board[38] <- "C"
board[39] <- "P"
board[40] <- "CDPST"

# Count how many states we have
statesIndex <- 1  #initial state (0,0)
statesCount <- 1
statesHolder <- matrix(0, boardLength * totalNumberOfBlues, 2)

for (i in 1:(boardLength - 1)) {
    # don't use last space, the 'coop'
    if (board[i] != "B") {
        bLi <- sum(blueSquares[1:i])  #blueSquares to Left of i, including i
        bRi <- sum(blueSquares[(i + 1):boardLength])  #blueSquares to Right of i
        
        maxChickens <- min(boardLength, i + bLi)
        minChickens <- max(0, i - bRi)
        
        statesCount <- statesCount + maxChickens - minChickens + 1
        for (j in minChickens:maxChickens) {
            statesIndex <- statesIndex + 1
            statesHolder[statesIndex, ] <- c(i, j)
        }
    }
}
statesHolder[statesCount + 1, ] <- c(Inf, Inf)  # win state
statesHolder[statesCount + 2, ] <- c(-Inf, -Inf)  # lose state
statesCount <- statesCount + 2  #add final won and loss states
states <- statesHolder[1:statesCount, ]
rm(statesHolder)

Q <- matrix(0, statesCount - 2, statesCount - 2)
R <- matrix(0, statesCount - 2, 2)
I <- diag(2)
Z <- matrix(0, 2, statesCount - 2)

for (i in 1:(statesCount - 2)) {
    currentSquare <- states[i, 1]
    chicksInCoop <- states[i, 2]
    currentState <- i
    
    # This sets transition probabilities from spinning a fox
    if (chicksInCoop == 0) {
        Q[i, i] <- 1/NAnimals  # stay in current state, prob 1/NAnimals
    } else if (states[i - 1, 1] == currentSquare) {
        Q[i, i - 1] <- 1/NAnimals  # remove 1 chick, fill trans prob
    } else {
        # too few chicks, must lose
        R[i, 2] <- R[i, 2] + 1/NAnimals  # so add to lose probability
    }
    
    # Now set transition probabilities from other five animals.
    for (animal in animalList) {
        
        nextSquare <- grep(animal, board[(currentSquare + 1):boardLength])[1] + currentSquare
        
        if (blueSquares[nextSquare] == 0) {
            # not Blue, just add advance
            updatedChicksInCoop <- chicksInCoop + (nextSquare - currentSquare)
            if (updatedChicksInCoop > boardLength) 
                updatedChicksInCoop <- boardLength
        } else {
            # Blue square, so add extra chick
            updatedChicksInCoop <- chicksInCoop + (nextSquare - currentSquare) + 
                1
            if (updatedChicksInCoop > boardLength) 
                updatedChicksInCoop <- boardLength
        }
        
        in_states <- FALSE
        
        for (p in 1:statesCount) {
            if (states[p, 1] == nextSquare && states[p, 2] == updatedChicksInCoop) {
                in_states <- TRUE
                newLocation <- p
                break
            }
        }
        
        if (in_states == TRUE) {
            Q[i, newLocation] <- 1/NAnimals
        } else {
            if (nextSquare == boardLength && updatedChicksInCoop == boardLength) 
                {
                  R[i, 1] <- R[i, 1] + 1/NAnimals
                }  # win
 else {
                R[i, 2] <- R[i, 2] + 1/NAnimals  #lose
            }
        }
    }
}

# Calculate win-lose probabilities with normal matrix: (I-Q)^{-1} R
IsC <- diag(statesCount - 2)
matrixProbs <- solve(IsC - Q, R)
\end{lstlisting}

\item[R] 

\link{http://www.math.unl.edu/~sdunbar1/MarkovChains/Games/risk.R}{R script for \emph{Risk}.}

  \begin{lstlisting}[language=R]
A <- 10
D  <- 10

totalStates <- A * D + (A + D)

stateToIndex <- function(a,d) {
    # given a state, retrun the index for the trans prob matrix
    if (a > 0) {
        ind <- (A+1 - a) + (D-d)*A  
    }  else {
        ind <- A*D + A + d 
    }
    ind
}

indexToState <- function(ind) {
    # given an index for a transient state, return the state as a pair
    if (ind <= A*D + A) {
        j <- A*D - ind
        r <- j %% A
        first <- 1 + r
        second <- 1 + (j-r)/A
    } else {
        first <- 0
        second <- ind - (A*D + A)
    }
    c(first, second)
}

pi111 <- 15/36                          # light gray
pi110 <- 21/36                          # magenta

pi121 <- 55/216                         # brown
pi120 <- 161/216                        # purple

pi211 <- 125/216                        # pink
pi210 <- 91/216                         # black

pi222 <- 295/1296                       # cyan
pi221 <- 420/1296                       # darkgreen
pi220 <- 581/1296                       # yellow

pi311 <- 855/1296                       # orange
pi310 <- 441/1296                       # gray

pi322 <- 2890/7776                      # green
pi321 <- 2611/7776                      # blue
pi320 <- 2275/7776                      # red

QR <- matrix(0, A * D, A * D + A + D)
## R <- matrix(0, A * D, A + D)
I <- diag(A + D)
Z <- matrix(0, A + D, A * D)

att3dice <- seq(from=A, to = 3, by=-1)
def2dice <- seq(from=D, to = 2, by=-1)

for (d in def2dice) {
    for (a in att3dice) {
        QR[stateToIndex(a,d), stateToIndex(a, d-2)] <- pi322   # green
        QR[stateToIndex(a,d), stateToIndex(a-1, d-1)] <- pi321 # blue
        QR[stateToIndex(a,d), stateToIndex(a-2, d)] <- pi320 # red
    }
    QR[stateToIndex(2, d), stateToIndex(2, d-2)] <- pi222 # cyan
    QR[stateToIndex(2, d), stateToIndex(1, d-1)] <- pi221 # darkgreen
    QR[stateToIndex(2, d), stateToIndex(0, d)] <- pi220 # yellow

    QR[stateToIndex(1, d), stateToIndex(1, d-1)] <- pi121 # purple
    QR[stateToIndex(1, d), stateToIndex(0, d)] <- pi120 # brown
}

for (a in att3dice) {
    QR[stateToIndex(a, 1), stateToIndex(a, 0)] <- pi311 # orange
    QR[stateToIndex(a, 1), stateToIndex(a-1, 1)] <- pi310 # gray
}

QR[stateToIndex(2, 1), stateToIndex(2, 0)] <- pi211 # pink
QR[stateToIndex(2, 1), stateToIndex(1, 1)] <- pi210 # black
QR[stateToIndex(1, 1), stateToIndex(1, 0)] <- pi111 # lightgray
QR[stateToIndex(1, 1), stateToIndex(0, 1)] <- pi110 # magenta


M <- rbind(QR, cbind(Z, I))

library("markovchain")

riskChain <- new("markovchain", states = as.character(1:totalStates), byrow = TRUE, 
    transitionMatrix = M, name = "RISK")
absorpProb <- absorptionProbabilities(riskChain)
attackerWins  <- apply(absorpProb[ , 1:A], 1, sum)
attackerWinTable  <- t(matrix(formatC(attackerWins, digits=3, format="f"), A,D))
attackerWinTable
\end{lstlisting}

% \item[Octave]                   %

% \link{http://www.math.unl.edu/~sdunbar1/    .m}{Octave script for .}

% \begin{lstlisting}[language=Octave]

% \end{lstlisting}

% \item[Perl] 

% \link{http://www.math.unl.edu/~sdunbar1/    .pl}{Perl PDL script for .}

% \begin{lstlisting}[language=Perl]

% \end{lstlisting}

% \item[SciPy] 

% \link{http://www.math.unl.edu/~sdunbar1/    .py}{Scientific Python script for .}

% \begin{lstlisting}[language=Python]

% \end{lstlisting}

\end{description}



%%% Local Variables:
%%% mode: latex
%%% TeX-master: t
%%% End:


\hr

\visual{Problems to Work}{../../../../CommonInformation/Lessons/solveproblems.png}
\section*{Problems to Work for Understanding}

\renewcommand{\theexerciseseries}{}
\renewcommand{\theexercise}{\arabic{exercise}}

\begin{exercise}
    The CDF and PDF functions \( P(j) \) and \( p(j) \) above
    respectively summarize the probability that a single player will get
    Tenzi in \( j \) rolls.  What if \( n \) people are playing?  This
    essentially taking \( n \) independent random samples from the
    probability distribution \( p(j) \), and identifying the minimum
    value(s) of \( j \) as the roll in which the game is won.  For
    example, if there are \( n=5 \) players, take \( 5 \) independent
    random samples from \( p(j) \).  Suppose the arbitrary values of \(
    j \) are \( [12,20,15,9,28] \), representing the number of rolls
    needed for each player to get Tenzi.  Then the \( 4 \)th player wins
    the game in \( 9 \) rolls.  In general, write an expression for the
    probability that the minimum value from \( n \) random samples of \(
    p(j) \) is less than or equal to \( j \) in terms of \( P(j) \).
\end{exercise}
\begin{solution}
    Denote as \( P(n,j) \) the probability that the minimum value from \(
    n \) random samples of \( p(j) \) is less than or equal to \( j \).
    Then \( P(n,j) \) is the complement of the probability that all \( n
    \) random samples (which are independent of each other) are greater
    than \( j \), so \( P(n,j)=1 - (1 - P(j))^n \).  The pdf giving the
    probability that a game with \( n \) players is won in exactly \( j \)
    rolls is \( p(n, j)=P(n,j) - P(n,j - 1) \).
\end{solution}
\begin{exercise}
    Write a script in some computer language to replicate the
    calculations and graphs for the Markov chain for \emph{Tenzi}.
\end{exercise}
\begin{solution}
    The transition probability matrix for \emph{Tenzi} (rounded to three
    decimal places) is written in Maxima as
\begin{verbatim}
pi[i,j] := if (j >= i) then  binomial(11-i, j-i) * ((1/6)^(j-i)) *
((5/6)^(11-j)) else 0;
P : genmatrix(pi, 11, 11);
fpprintprec : 3;
float(P);
\end{verbatim}
    Transfer this matrix of values to R and evaluate with the following
    script:
\begin{verbatim}
x  <-  c(0.1615055828898457, 0.3230111657796915, 0.2907100492017223,
         0.1550453595742519, 0.05426587585098816, 0.01302381020423716,
         0.002170635034039527, 2.48072575318803*10^-4,
         1.860544314891023*10^-5, 8.2690858439601*10^-7,
         1.65381716879202*10^-8, 0.0, 0.1938066994678149,
         0.3488520590420667, 0.2790816472336534, 0.1302381020423716,
         0.03907143061271148, 0.007814286122542296,
         0.001041904816338973, 8.930612711476907*10^-5,
         4.465306355738455*10^-6, 9.922903012752121*10^-8, 0.0, 0.0,
         0.2325680393613778, 0.3721088629782046, 0.2604762040847432,
         0.1041904816338973, 0.02604762040847432, 0.004167619265355891,
         4.167619265355891*10^-4, 2.381496723060509*10^-5,
         5.953741807651273*10^-7,0.0, 0.0, 0.0, 0.2790816472336534,
         0.3907143061271148, 0.2344285836762689, 0.07814286122542295,
         0.01562857224508459, 0.001875428669410151,
         1.250285779606767*10^-4, 3.572245084590763*10^-6,0.0, 0.0,
         0.0, 0.0, 0.3348979766803841, 0.4018775720164609,
         0.2009387860082305, 0.05358367626886145, 0.008037551440329218,
         6.430041152263375*10^-4, 2.143347050754458*10^-5,0.0, 0.0,
         0.0, 0.0, 0.0, 0.4018775720164609, 0.4018775720164609,
         0.1607510288065844, 0.03215020576131687, 0.003215020576131687,
         1.286008230452675*10^-4,0.0, 0.0, 0.0, 0.0, 0.0, 0.0,
         0.4822530864197531, 0.3858024691358025, 0.1157407407407407,
         0.0154320987654321, 7.716049382716049*10^-4,0.0, 0.0, 0.0,
         0.0, 0.0, 0.0, 0.0, 0.5787037037037037, 0.3472222222222222,
         0.06944444444444445, 0.004629629629629629,0.0, 0.0, 0.0, 0.0,
         0.0, 0.0, 0.0, 0.0, 0.6944444444444444, 0.2777777777777778,
         0.02777777777777778,0.0, 0.0, 0.0, 0.0, 0.0, 0.0, 0.0, 0.0,
         0.0, 0.8333333333333334, 0.1666666666666667,0.0, 0.0, 0.0,
         0.0, 0.0, 0.0, 0.0, 0.0, 0.0, 0.0, 1.0)
P  <-  t( matrix(x, 11, 11) )

PPow  <- P
tenzicdf  <- c(PPow[1,11])
for (i in 2:50) {
  PPow  <- PPow %*% P;
  tenzicdf  <- c(tenzicdf, PPow[1,11])
}

library(ggplot2)
ggplot(data = as.data.frame(tenzicdf), aes(x=1:50, y=tenzicdf))
  + geom_point()
  + geom_line()

tenzipdf  <-  tenzipdf[2:50] - tenzipdf[1:49]
tenzipdf  <- c(tenzicdf[1], tenzipdf)
ggplot(data = as.data.frame(tenzipdf), aes(x=1:50, y=tenzipdf))
    + geom_point()
    + geom_line()

sum(tenzipdf)
tenzimean  <- (1:50) %*% tenzipdf

Q  <-  P[1:10, 1:10]
I10  <-  ident(10)
N  <- solve( I10 - Q, I10)
N %*% rep(1,10)

Varw  <- ((2 * N - I10) %*% N) %*% rep(1, 10) - ( N %*% rep(1, 10) )^2
sdw  <- sqrt(Varw)
\end{verbatim}
\end{solution}

\begin{exercise}
    Show that the shortest possible game of \emph{Chutes and Ladders}
    consists of \( 7 \) moves by listing those moves.
\end{exercise}
\begin{solution}
    A possible path is to roll a \( 1 \), to \( 38 \); roll a \( 6 \),
    to \( 44 \); roll a \( 6 \), to \( 50 \); roll a \( 1 \), to \( 67 \);
    roll a \( 6 \), to \( 73 \); roll a \( 6 \), to \( 79 \) and
    finally, roll a \( 1 \) to ascend from \( 80 \) to \( 100 \).
    Examination of the first \( 6 \) powers of the transition
    probability matrix show that the probability of \( 6 \) step paths
    from \( 38 \) is \( 0.007265947 \) but \( 0 \) for fewer than \( 6 \)
    steps.

    Another possible path is roll a \( 4 \), to \( 14 \); roll a \( 6 \),
    to \( 20 \); roll a \( 6 \), to \( 26 \); roll a \( 2 \), to \( 84 \);
    roll a \( 6 \), to \( 90 \); roll a \( 5 \), to \( 75 \) and
    finally, roll a \( 5 \) to ascend from \( 80 \) to \( 100 \).
    Examination of the first \( 6 \) powers of the transition
    probability matrix show that the probability of \( 6 \) step paths
    from \( 14 \) is \( 0.001800412 \) but \( 0 \) for fewer than \( 6 \)
    steps.
\end{solution}

\begin{exercise}
    Consider the effect of chutes and ladders on the waiting time to
    absorption.
    \begin{enumerate}[label=(\alph*)]
    \item
        Find the expected waiting time to absorption in a game with no
        chutes or ladders.
    \item
        In the original game, insert an additional ladder from state 46
        to state 94.  What happens to the expected waiting time?
    \item
        In the original game, instead insert a chute from state 83 to
        state \( 7 \).  What happens to the expected waiting time?
    \item
        Add a ladder to the original game from state \( 79 \) to state \(
        81 \).  What happens to the expected waiting time?
    \item
        Add a chute to the original game from state \( 29 \) to state \(
        27 \).  What happens to the expected waiting time?
\end{enumerate}
games.tex.orig:1266:begin/end environment name mismatch
	begin {} at line 0: end {exercise} at line 1266
games.tex.orig:1266:negative environment level at this line: reset to zero
\end{exercise}
\begin{solution}
    \begin{enumerate}[label=(\alph*)]
    \item
        The expected length of a game with no chutes or ladders is \(
        35.68571 \).
    \item
        The expected length of the game drops from about \( 39.2 \) to
        about \( 29.8 \);
    \item
        The game lengthens to about \( 45.8 \) moves.
    \item
        Lengthens the expected playing time by more than two moves to
        about \( 41.9 \), since it increases the chances of missing the
        important ladder from state \( 80 \) to \( 100 \).
    \item
        Shortens the game by over a move to about \( 38.0 \), since it
        gives a second chance at the long ladder from \( 28 \) to \( 84 \).
\end{enumerate}
games.tex.orig:1285:begin/end environment name mismatch
	begin {} at line 0: end {solution} at line 1285
games.tex.orig:1285:negative environment level at this line: reset to zero
\end{solution}

\begin{exercise}
    Investigate the following modifications to the small \emph{Count
    Your Chickens!} game.
    \begin{enumerate}[label=(\alph*)]
    \item
        What is the win probability if only square \( 2 \) is blue? What
        is the win probability if squares \( 2 \), \( 3 \) and \( 5 \)
        are blue?
    \item
        What effect does adding additional animals to the board (i.e.,
        increasing the size of the spinner) have?
    \item
        What if the fox removes more than one chick from the coop?
    \item
        What effect does changing the size of each character's space in
        the spinner do?  For example, what if the fox covered \( 1/3 \)
        of the spinner and the remaining five animals each took up \(
        2/15 \) of the spinner?
    \item
        What if some of the blue squares were colored green and had the
        effect of moving an additional two chicks into the coop?
    \item
        Create your own questions that can be analyzed using Markov
        chains and to modify the supplemental R code to answer the
        questions.
\end{enumerate}
games.tex.orig:1313:begin/end environment name mismatch
	begin {} at line 0: end {exercise} at line 1313
games.tex.orig:1313:negative environment level at this line: reset to zero
\end{exercise}
\begin{solution}
    The win probability if only square \( 2 \) is blue s \( 0.5288 \)
    The win probability if squares \( 2 \), \( 3 \) and \( 5 \) are blue
    is \( 0.7761 \).  The remaining questions make a good project for
    student investigation.
\end{solution}

\hr

\visual{Books}{../../../../CommonInformation/Lessons/books.png}
\section*{Reading Suggestion:}

\bibliography{../../../../CommonInformation/bibliography}

%   \begin{enumerate}
%     \item
%     \item
%     \item
%   \end{enumerate}

\hr

\visual{Links}{../../../../CommonInformation/Lessons/chainlink.png}
\section*{Outside Readings and Links:}
\begin{enumerate}
    \item
    \item
    \item
    \item
\end{enumerate}

\section*{\solutionsname} \loadSolutions

\hr

\mydisclaim \myfooter

Last modified:  \flastmod

\end{document}

